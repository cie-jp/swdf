\documentclass[a4j]{jsarticle}

\usepackage{amsmath}
\usepackage{amssymb}
\usepackage{amsfonts}
\usepackage{bm}
\usepackage{graphicx}
\usepackage{color}
\usepackage{comment}
\usepackage{braket}

\usepackage{listings}
\usepackage{./style/jlisting}

\lstset{
language={C++},
backgroundcolor={\color[gray]{.98}},
basicstyle={\small},
identifierstyle={\small},
commentstyle={\small\ttfamily \color[rgb]{0,0.5,0}},
keywordstyle={\small\bfseries \color[rgb]{0.2,0.2,1.0}},
ndkeywordstyle={\small},
stringstyle={\small\ttfamily \color[rgb]{1,0.2,0.2}},
frame={tb},
breaklines=true,
columns=[l]{fullflexible},
numbers=left,
xrightmargin=0zw,
xleftmargin=3zw,
numberstyle={\scriptsize},
stepnumber=1,
numbersep=1zw,
morecomment=[l]{//}
}

\begin{document}
\section{メモ}

\subsubsection{上2重対角化}
\begin{lstlisting}[caption=上2重対角化,label=ほげ]
// void REAL__MATRIX_BIDIAGONALIZATION(REAL A[],REAL wu[],REAL wv[],INT row,INT col);

#define ROW 7
#define COL 5

REAL A[ROW * COL];// 入力行列
REAL wu[ROW];     // 重み1
REAL wv[COL];     // 重み2

REAL__MATRIX_BIDIAGONALIZATION(A,wu,wv,ROW,COL);// Aが変更され上2重対角化される. 

\end{lstlisting}

\subsubsection{行列式}
行列式の計算をどのように行うか?

2重対角行列
\begin{align}
\bm{B}&=
\begin{bmatrix}
a_{0}&b_{0}&0&\ldots &0&0&0\\
0&a_{1}&b_{1}&0&\ldots &0&0\\
0&0&a_{2}&b_{2}&0&\ldots &0\\
\end{bmatrix}
\end{align}

Householder変換により上2重対角化を行い, 対角成分の積を計算する. 
\begin{align}
\det \bm{B}=\prod _{n=1}^{N}a_{n}
\end{align}


\begin{lstlisting}[caption=上2重対角化,label=ほげ]


REAL A[ROW * COL];// 入力行列
REAL wu[ROW];     // 重み1
REAL wv[COL];     // 重み2

REAL__MATRIX_BIDIAGONALIZATION(A,wu,wv,ROW,COL);// Aが変更され上2重対角化される. 

\end{lstlisting}

%REAL__MATRIX_LU_DECOMPOSITION

\subsubsection{LU分解}
\subsubsection{特異値分解}
符号の計算. 
./include/CPP/TMatrix.h


%template<>
%REAL          CLDIA::det  (const TMatrix<REAL> &A  );


\section{TMatrix<REAL>}



\lstinputlisting[caption=TMatrixのサンプル,label=ラベル]{./srcs/tmatrix.cpp}

\end{document}
