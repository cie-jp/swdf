\documentclass[a4j]{jsarticle}

\usepackage{amsmath}
\usepackage{amssymb}
\usepackage{amsfonts}
\usepackage{bm}
\usepackage{graphicx}
\usepackage{color}
\usepackage{comment}
\usepackage{braket}

\usepackage{listings}
\usepackage{./style/jlisting}

\lstset{
language={C++},
backgroundcolor={\color[gray]{.98}},
basicstyle={\small},
identifierstyle={\small},
commentstyle={\small\ttfamily \color[rgb]{0,0.5,0}},
keywordstyle={\small\bfseries \color[rgb]{0.2,0.2,1.0}},
ndkeywordstyle={\small},
stringstyle={\small\ttfamily \color[rgb]{1,0.2,0.2}},
frame={tb},
breaklines=true,
columns=[l]{fullflexible},
numbers=left,
xrightmargin=0zw,
xleftmargin=3zw,
numberstyle={\scriptsize},
stepnumber=1,
numbersep=1zw,
morecomment=[l]{//}
}

\begin{document}
\section{メモ}
\begin{lstlisting}[caption=行列式メモ(現在の関数定義),label=ほげ]
REAL CLDIA::TMatrix<REAL>::det(); 
は行列式の絶対値を特異値分解により求める. 絶対値になっているので, 符号を考慮する際に注意が必要. 

REAL CLDIA::TMatrix<REAL>::det_lu(); 
はLU分解により, 
REAL CLDIA::TMatrix<REAL>::det_bidiag(); 
は, 2重対角化により行列式を求める. 符号を含めて正確に計算される. 
\end{lstlisting}

\begin{lstlisting}[caption=行列式メモ(最終的に目指す関数定義),label=ほげ]
// 行列式を計算する. 
void CLDIA::det(const TMatrix<REAL> &A);
REAL CLDIA::TMatrix<REAL>.det();

// 行列式の絶対値の自然対数を計算する. 
void CLDIA::ln_abs_det(const TMatrix<REAL> &A);
REAL CLDIA::TMatrix<REAL>.ln_abs_det();
\end{lstlisting}

\subsubsection{上2重対角化}
\begin{lstlisting}[caption=上2重対角化,label=ほげ]
// void REAL__MATRIX_BIDIAGONALIZATION(REAL A[],REAL wu[],REAL wv[],INT row,INT col);

#define ROW 7
#define COL 5

REAL A[ROW * COL];// 入力行列
REAL wu[ROW];     // 重み1
REAL wv[COL];     // 重み2

REAL__MATRIX_BIDIAGONALIZATION(A,wu,wv,ROW,COL);// Aが変更され上2重対角化される. 

\end{lstlisting}

\subsubsection{行列式}
行列式の計算をどのように行うか?

2重対角行列
\begin{align}
\bm{B}&=
\begin{bmatrix}
a_{0}&b_{0}&0&\ldots &0&0&0\\
0&a_{1}&b_{1}&0&\ldots &0&0\\
0&0&a_{2}&b_{2}&0&\ldots &0\\
\end{bmatrix}
\end{align}

Householder変換により上2重対角化を行い, 対角成分の積を計算する. 

正方行列を上2重対角化する場合, 
$(D-1)$回$A$の左から, $(D-2)$回右から行列式-1のHouseholder直交変換行列を作用させる必要がある. 
\begin{align}
\bm{B}&=\bm{U}_{D-1}\cdots \bm{U}_{2}\bm{U}_{1}\bm{A}\bm{V}_{1}^{\dag}\bm{V}_{2}^{\dag}\cdots \bm{V}_{D-2}^{\dag}
\end{align}
すなわち, 
\begin{align}
\bm{A}&=\bm{V}_{D-2}\cdots \bm{V}_{2}\bm{V}_{1}\bm{B}\bm{U}_{1}^{\dag}\bm{U}_{2}^{\dag}\cdots \bm{U}_{D-1}^{\dag}
\end{align}
となる. よって
\begin{align}
\det \bm{A}&=(-1)^{D-2}\times \det \bm{B}\times (-1)^{D-1}\nonumber\\
&=-\det \bm{B}\nonumber\\
&=-\prod _{d=1}^{D}b_{0d}
\end{align}

\subsection{Householder変換によるQR分解を用いた行列式の計算}

\begin{lstlisting}[caption=上2重対角化,label=ほげ]


REAL A[ROW * COL];// 入力行列
REAL wu[ROW];     // 重み1
REAL wv[COL];     // 重み2

REAL__MATRIX_BIDIAGONALIZATION(A,wu,wv,ROW,COL);// Aが変更され上2重対角化される. 

\end{lstlisting}

%REAL__MATRIX_LU_DECOMPOSITION

\subsubsection{LU分解}
\subsubsection{特異値分解}
符号の計算. 
./include/CPP/TMatrix.h


%template<>
%REAL          CLDIA::det  (const TMatrix<REAL> &A  );


\section{TMatrix<REAL>}



\lstinputlisting[caption=TMatrixのサンプル,label=ラベル]{./srcs/tmatrix.cpp}

\end{document}
